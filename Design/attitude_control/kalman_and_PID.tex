\documentclass[a4paper]{article}
\usepackage[german,ngerman]{babel}
\usepackage[T1]{fontenc}
\usepackage[utf8]{inputenc}
\usepackage{lmodern}
\usepackage{amsmath}
\usepackage{icomma}
\usepackage{tikz}
\usetikzlibrary{shapes,arrows,positioning}
\renewcommand{\u}[1]{\;\mathrm{#1}}
\newcommand{\uf}[1]{\;\mathrm{#1}}
\begin{document}

\tikzstyle{block} = [draw, rectangle, minimum height=2em, minimum width=2em]
\tikzstyle{sum} = [draw, circle]
\tikzstyle{input} = [coordinate]
\tikzstyle{output} = [coordinate]
\tikzstyle{pinstyle} = [pin edge={to-,thin,black}]

\begin{tikzpicture}[auto,node distance=2cm,>=latex']

	\node [input] (input) {};
	\node [sum, right of=input] (sum_x) {};
	\node [block, right of=sum_x] (PID) {PID};
	\node [block, right of=PID] (rocket) {system};
	\node [output, right of=rocket] (output) {};
	
	\draw [->] (input) -- node {$x_0$} node [pos=0.9] {$+$} (sum_x);
	\draw [->] (sum_x) -- node {$e$} (PID);
	\draw [->] (PID) -- node {$u$} (rocket);
	\draw [->] (rocket) -- node [near start] {$x$} (output);
	
	\draw node at (7,0) {\textbullet};
	\draw [->] (7,0) |- (2, -1) -| node [pos=0.90] {$-$} (sum_x);
	\draw node at (6,1) {process noise $w$};
	\draw [->] (6,0.8) -| (rocket);
	
\end{tikzpicture}

\begin{tikzpicture}[auto,node distance=2cm,>=latex']

	\node [input] (input) {};
	\node [sum, right of=input] (sum_x) {};
	\node [block, right of=sum_x] (PID) {PID};
	\node [block, right of=PID] (servo) {servo};
	\node [block, right of=servo] (rocket) {rocket};
	\node [output, right of=rocket] (output) {};
	
	\draw [->] (input) -- node {$x_0$} node [pos=0.9] {$+$} (sum_x);
	\draw [->] (sum_x) -- node {$e$} (PID);
	\draw [->] (PID) -- node {$u$} (servo);
	\draw [->] (servo) -- node {$M$} (rocket);
	\draw [->] (rocket) -- node [near start] {$x$} (output);
	
	\draw node at (9,0) {\textbullet};
	\draw [->] (9,0) |- (2, -1) -| node [pos=0.90] {$-$} (sum_x);
	\draw node at (8,1) {process noise $w$};
	\draw [->] (8,0.8) -| (rocket);
	
\end{tikzpicture}

\begin{tikzpicture}[auto,node distance=2cm,>=latex']

%	\draw[help lines] (0,-2) grid (8,3);

	\node [input] (input) {};
	\node [block, minimum width=3cm, right of=input, align=center] at (0.5,0) (system) {system \\ $\dot{X} = AX+w$ \\ $y= CX$};
	\node [sum, right=1.3cm of system] (sum_y) {};
	\node [block, right of=sum_y, align=center] (kalman) {Kalman\\filter};
	\node [output, right of=kalman] (output) {};
	
	\draw [->] (system) -- node[pos=0.8] {$+$} (sum_y);
	\draw [->] (sum_y) -- node {$\tilde{y}$} (kalman);

	\draw node at (2.5,1.5) {process noise $w$};
	\draw [->] (2.5,1.3) -| (system);

	\draw node at (5.5,1.5) {signal noise $v$};
	\draw [->] (5.5,1.3) -| node[pos=0.9] {$+$} (sum_y);
	
	\draw [->] (kalman) -- node {$\hat{X}$} (output);


\end{tikzpicture}

\begin{tikzpicture}[auto,node distance=2cm,>=latex']

	\node [input] (input) {};
	\node [sum, right of=input] (sum_x) {};
	\node [block, right of=sum_x] (PID) {PID};
	\node [block, right of=PID] (servo) {servo};
	\node [block, right of=servo] (rocket) {rocket};
	\node [sum, right of=rocket] (sum_y) {};
	\node [block, minimum height=1.5cm, minimum width=2cm, right of=sum_y, align=center] (kalman) {Kalman\\filter};
	\node [output, right of=kalman] (output) {};
	
	\draw [->] (input) -- node {$x_0$} node [pos=0.9] {$+$} (sum_x);
	\draw [->] (sum_x) -- node {$\hat{e}$} (PID);
	\draw [->] (PID) -- node {$u$} (servo);
	\draw [->] (servo) -- node {$M$} (rocket);
	\draw [->] (rocket) -- node[pos=0.8] {$+$} (sum_y);
	\draw [->] (sum_y) -- node {$\tilde{y}$} (kalman);
	\draw [->] (kalman) -- node {$\hat{X}$} (output);
	\draw node at (5,0) {\textbullet};
	\draw [->] (5,0) |- (10.5,-1) -| (10.5,-0.5) -- (11,-0.5);
	\draw node at (13.5,0) {\textbullet};
	\draw [->] (13.5,0) |- (2,-1.5) -| node [pos=0.9] {$-$} (sum_x);
	
%	\node [block, minimum width=3cm, right=1.3cm of=sum_x, align=center] at (0.5,0) (system) {system \\ $\dot{X} = AX+w$ \\ $y= CX$};

\end{tikzpicture}

\Large{HIER WERDEN KONTINUIERLICHE UND DISKRETE ZUSTANDSGLEICHUNGEN VERMISCHT!}

\[ \dot{X} = AX \]
ist nicht
\[ X_{i+1} = AX_i \]!

Zustandsvektor für die Rakete
\[ X_{\textrm{rocket}} = \begin{pmatrix}\gamma \\ \dot{\gamma} \\ \ddot{\gamma} \end{pmatrix} \]
Zustandsgleichung
\[ \dot{X}_{\textrm{rocket}} = \begin{pmatrix} 
1 & \Delta t & 0 \\ 
0 & 1 & \Delta t \\ 
0 & 0 & 1 
\end{pmatrix} X_{\textrm{rocket}} \]
Differenzialgleichung die das Verhalten des Servos beschreibt
\[ 0,000088 \ddot{\varphi} + 0,019 \dot{\varphi} + \varphi = u \]
Mit dem Zustandsvektor $X_{\textrm{servo}} = \begin{pmatrix} \varphi \\ \dot{\varphi} \end{pmatrix}$ ergibt sich für die Zustandsgleichung des Servos
\[ \dot{X}_{\textrm{servo}} = \begin{pmatrix} 0 & 1 \\ -11364 & -216 \end{pmatrix} X_{\textrm{servo}} + \begin{pmatrix} 0 \\ 11643 \end{pmatrix} u \]
Zusätzlich bewirkt die Drehung des Servos über den Schub, den Hebelarm und die Trägheit der Rakete eine Beschleunigung der Rakete um eine Querachse
\[ \ddot{\gamma} = \frac{F a}{J} \varphi \]
Mit einem Zustandsvektor der Rakete und Servo abdeckt
\[ X_{\textrm{system}} = \begin{pmatrix} \gamma \\ \dot{\gamma} \\ \ddot{\gamma} \\ \varphi \\ \dot{\varphi} \end{pmatrix} \]
erhalten wir die Zustandsgleichung
\[ X_{\textrm{system}} = \begin{pmatrix} 1 & \Delta t & 0 & 0 & 0 \\
0 & 1 & \Delta t & 0 & 0 \\
0 & 0 & 1 & \frac{F a}{J} & 0 \\
0 & 0 & 0 & 0 & 1 \\
0 & 0 & 0 & -11364 & -216 \end{pmatrix} X_{\textrm{system}} +\begin{pmatrix} 0 \\ 0 \\ 0 \\ 0 \\ 11643 \end{pmatrix} u \]

\end{document}
